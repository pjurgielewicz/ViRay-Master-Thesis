% Please add the following required packages to your document preamble:
% \usepackage{lscape}
% \usepackage{longtable}
% Note: It may be necessary to compile the document several times to get a multi-page table to line up properly
\begin{landscape}
\begin{longtable}[c]{|r|l|l|l|}
\caption[Architektura instrukcji stworzonego układu procesora]{Architektura instrukcji stworzonego układu procesora. Za każdym razem, gdy w tabeli używane jest \texttt{reg} następuje dostęp do tablicy wektorowych rejestrów wewnętrznych, zaś gdy \texttt{mem} dochodzi do dostępu do pamięci zewnętrznej. Warunki nakładane na instrukcje (\texttt{i}) dotyczą tego, dla jakich numerów podinstrukcji można je stosować}
\label{ch3:tab:instructions}\\
\hline
\multicolumn{1}{|c|}{\textbf{Instrukcja}} & \multicolumn{1}{c|}{\textbf{Kod}} & \multicolumn{1}{c|}{\textbf{Operacja}}                                                                                                      & \multicolumn{1}{c|}{\textbf{Uwagi}}                                                                                                                                                                                                                                                                                                                                            \\ \hline

\endfirsthead
%
\multicolumn{4}{c}%
{{\bfseries \bfseries Tablica \thetable\ : kontynuacja}} \\

\endhead
%
\textbf{NOP}                              & \textit{0000}                     & -                                                                                                                                           & Instrukcja pusta                                                                                                                                                                                                                                                                                                                                                               \\ \hline
\textbf{LC}                               & \textit{0001}                     & reg{[}wr{]}{[}i{]} = val                                                                                                                    &                                                                                                                                                                                                                                                                                                                                                                                \\ \hline
\textbf{LV}                               & \textit{0010}                     & reg{[}wr{]} = mem{[}val{]}                                                                                                                  & \begin{tabular}[c]{@{}l@{}}$\mathtt{i\in\left\lbrace 2, 3 \right\rbrace }$\\ Wykonuje się przed innymi podinstrukcjami. \\ Efekt odczytu może zostać użyty przez inną podinstrukcję w tej\\ samej instrukcji\end{tabular}                                                                                                                                                      \\ \hline
\textbf{LVI}                              & \textit{0011}                     & reg{[}wr{]} = mem{[}reg{[}r0{]}{[}idx0{]}{]}                                                                                                & \begin{tabular}[c]{@{}l@{}}$\mathtt{i=3}$\\ Pozostałe podinstrukcje w danej instrukcji nie zostaną wykonane\end{tabular}                                                                                                                                                                                                                                                       \\ \hline
\textbf{MOV}                              & \textit{0100}                     & reg{[}wr{]}{[}i{]} = reg{[}r0{]}{[}idx0{]}                                                                                                  &                                                                                                                                                                                                                                                                                                                                                                                \\ \hline
\textbf{ADD}                              & \textit{0101}                     & reg{[}wr{]}{[}i{]} = reg{[}r0{]}{[}idx0{]} + reg{[}r1{]}{[}idx1{]}                                                                          &                                                                                                                                                                                                                                                                                                                                                                                \\ \hline
\textbf{SHR}                              & \textit{0110}                     & reg{[}wr{]}{[}i{]} = reg{[}r0{]}{[}idx0{]} \textgreater{}\textgreater r1                                                                    &                                                                                                                                                                                                                                                                                                                                                                                \\ \hline
\textbf{MUL}                              & \textit{0111}                     & reg{[}wr{]}{[}i{]} = reg{[}r0{]}{[}idx0{]} $\cdot$ reg{[}r1{]}{[}idx1{]}                                                                    &                                                                                                                                                                                                                                                                                                                                                                                \\ \hline
\textbf{DOT}                              & \textit{1000}                     & reg{[}wr{]}{[}i{]} = reg{[}r0{]} $\cdot$ reg{[}r1{]}                                                                                        & \begin{tabular}[c]{@{}l@{}}Iloczyn skalarny rejestrów wektorowych zredukowanych \\ do ich 3 pierwszych elementów\end{tabular}                                                                                                                                                                                                                                                  \\ \hline
\textbf{ADOT}                             & \textit{1001}                     & reg{[}wr{]}{[}i{]} += reg{[}r0{]} $\cdot$ reg{[}r1{]}                                                                                       & \begin{tabular}[c]{@{}l@{}}Iloczyn skalarny rejestrów wektorowych zredukowanych \\ do ich 3 pierwszych elementów\end{tabular}                                                                                                                                                                                                                                                  \\ \hline
\textbf{MADD}                             & \textit{1010}                     & reg{[}wr{]}{[}i{]} += reg{[}r0{]}{[}idx0{]} $\cdot$ reg{[}r1{]}{[}idx1{]}                                                                   &                                                                                                                                                                                                                                                                                                                                                                                \\ \hline
\textbf{PRE\_S}                           & \textit{1011}                     & \begin{tabular}[c]{@{}l@{}}reg{[}wr{]}{[}0{]} = fun1(reg{[}r0{]}{[}idx0{]})\\ reg{[}wr{]}{[}1{]} = fun2(reg{[}r0{]}{[}idx0{]})\end{tabular} & \begin{tabular}[c]{@{}l@{}}Instrukcja obliczająca wstępne parametry potrzebne,\\ aby iteracyjnie obliczyć odwrotność pierwiastka kwadratowego\end{tabular}                                                                                                                                                                                                                     \\ \hline
\textbf{PRE\_D}                           & \textit{1100}                     & \begin{tabular}[c]{@{}l@{}}reg{[}wr{]}{[}0{]} = fun1(reg{[}r0{]}{[}idx1{]})\\ reg{[}wr{]}{[}1{]} = fun2(reg{[}r0{]}{[}idx0{]})\end{tabular} & \begin{tabular}[c]{@{}l@{}}Instrukcja obliczająca wstępne parametry potrzebne,\\ aby iteracyjnie obliczyć iloraz dwóch liczb\end{tabular}                                                                                                                                                                                                                                      \\ \hline
\textbf{JMP}                              & \textit{1101}                     &                                                                                                                                             & \begin{tabular}[c]{@{}l@{}}$\mathtt{i = 3}$\\ Opuszcza wykonanie $\mathtt{unsigned(val)}$ kolejnych instrukcji\end{tabular}                                                                                                                                                                                                                                                    \\ \hline
\textbf{JMP\_IF}                          & \textit{1110}                     & \begin{tabular}[c]{@{}l@{}}idx0 == 0: res{[}idx1{]} == 0\\ idx0 == 1: res{[}idx1{]} \textgreater 0\end{tabular}                             & \begin{tabular}[c]{@{}l@{}}$\mathtt{i = 3}$\\ $\mathtt{res}$ - tablica przechowująca wyniki poprzedzających ją podinstrukcji\\ Opuszcza wykonanie $\mathtt{unsigned(val)}$ kolejnych instrukcji, jeśli warunek\\ uzależniony od $\mathtt{idx0}$ i wyniku konkretnej podinstrukcji wskazywanej \\ przez $\mathtt{idx1}$ $\mathrm{\underline{jest}}$ spełniony\end{tabular}      \\ \hline
\textbf{JMP\_IFN}                         & \textit{1111}                     & \begin{tabular}[c]{@{}l@{}}idx0 == 0: res{[}idx1{]} == 0\\ idx0 == 1: res{[}idx1{]} \textgreater 0\end{tabular}                             & \begin{tabular}[c]{@{}l@{}}$\mathtt{i = 3}$\\ $\mathtt{res}$ - tablica przechowująca wyniki poprzedzających ją podinstrukcji\\ Opuszcza wykonanie $\mathtt{unsigned(val)}$ kolejnych instrukcji, jeśli warunek\\ uzależniony od $\mathtt{idx0}$ i wyniku konkretnej podinstrukcji wskazywanej \\ przez $\mathtt{idx1}$ $\mathrm{\underline{nie\ jest}}$ spełniony\end{tabular} \\ \hline
\end{longtable}
\end{landscape}