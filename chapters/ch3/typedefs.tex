% Please add the following required packages to your document preamble:
% \usepackage{lscape}
% \usepackage{longtable}
% Note: It may be necessary to compile the document several times to get a multi-page table to line up properly
\begin{landscape}
\begin{longtable}[c]{|r|c|l|}
\caption[Opis najważniejszych parametrów sterujących zachowaniem modułu \textsc{ViRay}]{Opis najważniejszych parametrów sterujących zachowaniem modułu \textsc{ViRay} zdefiniowanych w pliku \texttt{typedefs.h}. Wartości domyślne odpowiadają finalnej, zoptymalizowanej wersji projektu implementowalnego na płytce ewaluacyjnej KCU116 z użyciem Vivado Design Suite 2017.4}
\label{ch3:tab:typedefs}\\
\hline
\multicolumn{1}{|c|}{\textbf{Definicja}}                                        & \textbf{Wartość domyślna}            & \multicolumn{1}{c|}{\textbf{Opis}}                                                                                                                                                                                                                                                                                                                    \\ \hline
\endfirsthead
%
\multicolumn{3}{c}%
{{\bfseries Tablica \thetable\ : kontynuacja}} \\
\hline
\multicolumn{1}{|c|}{\textbf{Definicja}}                                        & \textbf{Wartość domyślna}            & \multicolumn{1}{c|}{\textbf{Opis}}                                                                                                                                                                                                                                                                                                                    \\ \hline
\endhead
%
\textbf{$\mathrm{\textbf{\scriptsize{UC\_OPERATION}}}$}                         & \textit{(niezdefiniowane)}           & \begin{tabular}[c]{@{}l@{}}Musi zostać zdefiniowane w przypadku wykorzystywania\\ źródeł systemu śledzenia promieni przez \\ mikrokontroler (Microblaze)\end{tabular}                                                                                                                                                                                 \\ \hline
\textbf{$\mathrm{\textbf{\scriptsize{DESIRED\_INNER\_LOOP\_II}}}$}              & $\mathtt{8}$                         & \begin{tabular}[c]{@{}l@{}}Żądana przez użytkownika ilość cykli zegara (interwał) \\ pomiędzy obliczeniem kolejnych pikseli obrazu.\end{tabular}                                                                                                                                                                                                      \\ \hline
\textbf{$\mathrm{\textbf{\scriptsize{USE\_FLOAT}}}$}                            & \textit{(zdefiniowane)}              & \begin{tabular}[c]{@{}l@{}}Definiuje typ danych używany w trakcie obliczeń na:\\ $\mathtt{typedef\ float\ myType}$\end{tabular}                                                                                                                                                                                                                       \\ \hline
\textbf{$\mathrm{\textbf{\scriptsize{CORE\_BIAS}}}$}                            & $\mathtt{(myType(0.001))}$           & \begin{tabular}[c]{@{}l@{}}Niewielka dodatnia wartość rzeczywista pozwalająca \\ uniknąć artefaktów wizualnych związanych \\ z fałszywie pozytywnym wynikiem przecięcia, \\ a wynikających ze skończonej dokładności obliczeń\\ użytego typu danych\end{tabular}                                                                                      \\ \hline
\textbf{$\mathrm{\textbf{\scriptsize{MAX\_DISTANCE}}}$}                         & $\mathtt{(myType(1000000))}$         & \begin{tabular}[c]{@{}l@{}}Określa wartość przyjmowaną przez system śledzenia \\ promieni za nieskończoność\end{tabular}                                                                                                                                                                                                                              \\ \hline
\textbf{$\mathrm{\textbf{\scriptsize{SIMPLE\_OBJECT\_TRANSFORM\_ENABLE}}}$}     & \textit{(zdefiniowane)}              & \begin{tabular}[c]{@{}l@{}}Jeśli zdefiniowane, system umożliwia jedynie uproszczoną \\ manipulację obiektami w przestrzeni opartą o wartość skali \\ oraz przesunięcia względem początku globalnego układu \\ współrzędnych. \\ W przeciwnym razie realizowane są transformacje oparte \\ o przekształcenia macierzowe\end{tabular}                   \\ \hline
\textbf{$\mathrm{\textbf{\scriptsize{SELF\_RESTART\_ENABLE}}}$}                 & \textit{(niezdefiniowane)}           & \begin{tabular}[c]{@{}l@{}}Jeśli zdefiniowane, raz uruchomiony układ będzie generował \\ obrazy w nieskończoność. Raz odczytane dane tekstur \\ są przechowywane - redukuje to czas potrzebny \\ na inicjalizację pamięci tekstur (o ok. 65500 cykli), \\ jednak uniemożliwia animowanie danych tekstury.\end{tabular}                                \\ \hline
\textbf{$\mathrm{\textbf{\scriptsize{PIXEL\_COLOR\_CONVERSION\_ENABLE}}}$}      & \textit{(zdefiniowane)}              & \begin{tabular}[c]{@{}l@{}}Dokonuje transformacji zapisu koloru do bufora ramki \\ w taki sposób, aby możliwe było jego bezpośrednie użycie \\ na układzie KCU116. \\ Aby uzyskać poprawne kolory w środowisku testowym HLS,\\ wymagane jest wyłączenie tej opcji\end{tabular}                                                                        \\ \hline
\textbf{$\mathrm{\textbf{\scriptsize{RENDER\_DATAFLOW\_ENABLE}}}$}              & \textit{(zdefiniowane)}              & \begin{tabular}[c]{@{}l@{}}Jeśli zdefiniowane, dokonuje implementacji głównej pętli \\ przetwarzania z użyciem zapisu koloru pikseli \\ do tymczasowego bufora. \\ W przeciwnym wypadku piksele zapisywane \\ są w trybie ciągłym do bufora ramki.\end{tabular}                                                                                       \\ \hline
\textbf{$\mathrm{\textbf{\tiny{AGGRESSIVE\_AREA\_OPTIMIZATION\_ENABLE}}}$}      & \textit{(niezdefiniowane)}              & \begin{tabular}[c]{@{}l@{}}Umożliwia wykorzystanie typu $\mathtt{half}$ w obliczeniach \\ niewymagających wysokiej precyzji\end{tabular}                                                                                                                                                                                                              \\ \hline
\textbf{$\mathrm{\textbf{\scriptsize{SPHERE\_OBJECT\_ENABLE}}}$}                & \textit{(zdefiniowane)}              & Dodaje możliwość użycia sfer w scenie                                                                                                                                                                                                                                                                                                                 \\ \hline
\textbf{$\mathrm{\textbf{\scriptsize{PLANE\_OBJECT\_ENABLE}}}$}                 & \textit{(zdefiniowane)}              & Dodaje możliwość użycia płaszczyzn w scenie                                                                                                                                                                                                                                                                                                           \\ \hline
\textbf{$\mathrm{\textbf{\scriptsize{DISK\_OBJECT\_ENABLE}}}$}                  & \textit{(zdefiniowane)}              & Dodaje możliwość użycia dysków w scenie                                                                                                                                                                                                                                                                                                               \\ \hline
\textbf{$\mathrm{\textbf{\scriptsize{SQUARE\_OBJECT\_ENABLE}}}$}                & \textit{(zdefiniowane)}              & Dodaje możliwość użycia kwadratów w scenie                                                                                                                                                                                                                                                                                                            \\ \hline
\textbf{$\mathrm{\textbf{\scriptsize{CYLINDER\_OBJECT\_ENABLE}}}$}              & \textit{(zdefiniowane)}              & Dodaje możliwość użycia walców w scenie                                                                                                                                                                                                                                                                                                               \\ \hline
\textbf{$\mathrm{\textbf{\scriptsize{CUBE\_OBJECT\_ENABLE}}}$}                  & \textit{(niezdefiniowane)}           & Dodaje możliwość użycia sześcianów w scenie                                                                                                                                                                                                                                                                                                           \\ \hline
\textbf{$\mathrm{\textbf{\scriptsize{CONE\_OBJECT\_ENABLE}}}$}                  & \textit{(zdefiniowane)}              & Dodaje możliwość użycia stożków w scenie                                                                                                                                                                                                                                                                                                              \\ \hline
\textbf{$\mathrm{\textbf{\scriptsize{AMBIENT\_COLOR\_ENABLE}}}$}                & \textit{(zdefiniowane)}              & Światło otoczenia jako element oświetlenia powierzchni                                                                                                                                                                                                                                                                                                \\ \hline
\textbf{$\mathrm{\textbf{\scriptsize{DIFFUSE\_COLOR\_ENABLE}}}$}                & \textit{(zdefiniowane)}              & Światło rozproszone jako element oświetlenia powierzchni                                                                                                                                                                                                                                                                                              \\ \hline
\textbf{$\mathrm{\textbf{\scriptsize{SPECULAR\_HIGHLIGHT\_ENABLE}}}$}           & \textit{(zdefiniowane)}              & \begin{tabular}[c]{@{}l@{}}Światło odbite zwierciadlanie jako element oświetlenia \\ powierzchni\end{tabular}                                                                                                                                                                                                                                         \\ \hline
\textbf{$\mathrm{\textbf{\tiny{CUSTOM\_COLOR\_SPECULAR\_HIGHLIGHTS\_ENABLE}}}$} & \textit{(zdefiniowane)}              & \begin{tabular}[c]{@{}l@{}}Umożliwia dokonanie modyfikacji koloru światła odbitego \\ w sposób zwierciadlany za pomocą parametru materiału\end{tabular}                                                                                                                                                                                               \\ \hline
\textbf{$\mathrm{\textbf{\scriptsize{INTERNAL\_SHADING\_ENABLE}}}$}             & \textit{(zdefiniowane)}              & \begin{tabular}[c]{@{}l@{}}Dokonuje inwersji normalnej $\vec{n}$ do powierzchni w punkcie\\ zderzenia, w przypadku gdy iloczyn skalarny kierunku \\ promienia padającego $\vec{d}$ i normalnej $\vec{n}$ w punkcie \\ zderzenia jest nieujemny. \\ Umożliwia obrazowanie obu stron danej powierzchni\end{tabular}                                     \\ \hline
\textbf{$\mathrm{\textbf{\scriptsize{SHADOW\_ENABLE}}}$}                        & \textit{(zdefiniowane)}              & Włącza rzucanie cieni przez obiekty znajdujące się w świecie                                                                                                                                                                                                                                                                                          \\ \hline
\textbf{$\mathrm{\textbf{\scriptsize{SELF\_SHADOW\_ENABLE}}}$}                  & \textit{(zdefiniowane)}              & \begin{tabular}[c]{@{}l@{}}Włącza możliwość rzucania cieni przez dany obiekt \\ na samego siebie\end{tabular}                                                                                                                                                                                                                                         \\ \hline
\textbf{$\mathrm{\textbf{\scriptsize{FRESNEL\_REFLECTION\_ENABLE}}}$}           & \textit{(zdefiniowane)}              & \begin{tabular}[c]{@{}l@{}}Jeśli zdefiniowane, współczynnik odbicia określany jest \\ na podstawie wzorów Fresnela $\mathrm{\eqref{ch1:eq:SimpleFresnelRperp}\eqref{ch1:eq:SimpleFresnelRpara}}$. \\ W przeciwnym razie, jako współczynnik odbicia zawsze \\ brany będzie współczynnik odbicia zwierciadlanego $k_s$ \\ danego materiału\end{tabular} \\ \hline
\textbf{$\mathrm{\textbf{\scriptsize{OREN\_NAYAR\_DIFFUSE\_MODEL\_ENABLE}}}$}   & \textit{(zdefiniowane)}           & \begin{tabular}[c]{@{}l@{}}Włącza możliwość użycia modelu Orena-Nayara \\ dla modelowania światła rozproszonego\end{tabular}                                                                                                                                                                                                                          \\ \hline
\textbf{$\mathrm{\textbf{\tiny{TORRANCE\_SPARROW\_SPECULAR\_MODEL\_ENABLE}}}$}  & \textit{(zdefiniowane)}              & \begin{tabular}[c]{@{}l@{}}Włącza możliwość użycia modelu Torrance'a-Sparrowa \\ dla modelowania światła odbitego zwierciadlanie\end{tabular}                                                                                                                                                                                                         \\ \hline
\textbf{$\mathrm{\textbf{\scriptsize{TEXTURE\_ENABLE}}}$}                       & \textit{(zdefiniowane)}              & Umożliwia teksturowanie powierzchni                                                                                                                                                                                                                                                                                                                   \\ \hline
\textbf{$\mathrm{\textbf{\scriptsize{TEXTURE\_URAM\_STORAGE}}}$}                & \textit{(niezdefiniowane)}           & \begin{tabular}[c]{@{}l@{}}Jeśli zdefiniowane, instruuje narzędzia Vivado, \\ aby do przechowywania pamięci tekstury\\  wykorzystywane były bloki Ultra RAM. \\ W przeciwnym razie, tekstury zapisane będą \\ standardowo w BRAM\end{tabular}                                                                                                         \\ \hline
\textbf{$\mathrm{\textbf{\scriptsize{BILINEAR\_TEXTURE\_FILTERING\_ENABLE}}}$}  & \textit{(zdefiniowane)}              & \begin{tabular}[c]{@{}l@{}}Umożliwia dokonanie filtracji tekstury dla uzyskania \\ gładkich efektów przejścia między tekselami\end{tabular}                                                                                                                                                                                                           \\ \hline
\textbf{$\mathrm{\textbf{\scriptsize{ADVANCED\_TEXTURE\_MAPPING\_ENABLE}}}$}    & \textit{(zdefiniowane)}              & \begin{tabular}[c]{@{}l@{}}Umożliwia poprawne teksturowanie powierzchni \\ sferycznych i cylindrycznych\end{tabular}                                                                                                                                                                                                                                  \\ \hline
\textbf{$\mathrm{\textbf{\scriptsize{FAST\_INV\_SQRT\_ENABLE}}}$}               & \textit{(niezdefiniowane)}           & \begin{tabular}[c]{@{}l@{}}Stosuje szybki algorytm obliczania odwrotności  \\ pierwiastka kwadratowego. \\ Używane w środowisku testowym dla zmniejszenia czasu \\ oczekiwania na wyniki\end{tabular}                                                                                                                                                              \\ \hline
\textbf{$\mathrm{\textbf{\scriptsize{FAST\_DIVISION\_ENABLE}}}$}                & \textit{(niezdefiniowane)}           & \begin{tabular}[c]{@{}l@{}}Stosuje szybki algorytm obliczania wyniku dzielenia. \\ Używane w środowisku testowym dla zmniejszenia czasu \\ oczekiwania na wyniki\end{tabular}                                                                                                                                                                         \\ \hline
\textbf{$\mathrm{\textbf{\scriptsize{FAST\_ATAN2\_ENABLE}}}$}                   & \textit{(zdefiniowane)}              & \begin{tabular}[c]{@{}l@{}}Jeśli włączone, używa aproksymacji wielomianowej w celu \\ obliczenia wartości funkcji $\mathtt{atan2()}$. \\ W przeciwnym razie używana jest jej biblioteczna \\ dokładna wersja $\mathtt{hls::atan2()}$\end{tabular}                                                                                                     \\ \hline
\textbf{$\mathrm{\textbf{\scriptsize{FAST\_ACOS\_ENABLE}}}$}                    & \textit{(zdefiniowane)}              & \begin{tabular}[c]{@{}l@{}}Jeśli włączone, używa aproksymacji funkcji $\mathtt{acos()}$ \\ opartej o tablicę przeglądową i liniową interpolację między \\ odpowiednimi jej wartościami. \\ W przeciwnym razie używana jest biblioteczna \\ dokładna wersja $\mathtt{hls::atan2()}$\end{tabular}                                                       \\ \hline
\textbf{$\mathrm{\textbf{\scriptsize{WIDTH}}}$}                                 & $\mathtt{((unsigned\ short)(1920))}$ & Szerokość w pikselach ramki obrazu                                                                                                                                                                                                                                                                                                                    \\ \hline
\textbf{$\mathrm{\textbf{\scriptsize{HEIGHT}}}$}                                & $\mathtt{((unsigned\ short)(1080))}$ & Wysokość w pikselach ramki obrazu                                                                                                                                                                                                                                                                                                                     \\ \hline
\textbf{$\mathrm{\textbf{\scriptsize{WIDTH\_INV}}}$}                            & \textit{(automatyczne)}              & Odwrotność szerokości ramki obrazu                                                                                                                                                                                                                                                                                                                    \\ \hline
\textbf{$\mathrm{\textbf{\scriptsize{HEIGHT\_INV}}}$}                           & \textit{(automatyczne)}              & Odwrotność wysokości ramki obrazu                                                                                                                                                                                                                                                                                                                     \\ \hline
\textbf{$\mathrm{\textbf{\scriptsize{FRAME\_ROWS\_IN\_BUFFER}}}$}               & $\mathtt{((unsigned\ short)(20))}$   & \begin{tabular}[c]{@{}l@{}}Definiuje, ile pełnych wierszy obrazu będzie \\ przechowywanych w tymczasowym buforze koloru \\ (jeśli $\mathbf{RENDER\_DATAFLOW\_ENABLE}$ jest \\ zdefiniowane). \\ Wartość ta $\mathrm{\underline{musi}}$ być dzielnikiem $\mathbf{HEIGHT}$\end{tabular}                                                                 \\ \hline
\textbf{$\mathrm{\textbf{\scriptsize{VERTICAL\_PARTS\_OF\_FRAME}}}$}            & \textit{(automatyczne)}              & \begin{tabular}[c]{@{}l@{}}Opisuje, ile razy tymczasowy bufor koloru będzie musiał \\ zostać wypełniony, zanim nastąpi wygenerowanie pełnej \\ ramki obrazu\end{tabular}                                                                                                                                                                              \\ \hline
\textbf{$\mathrm{\textbf{\scriptsize{FRAME\_ROW\_BUFFER\_SIZE}}}$}              & \textit{(automatyczne)}              & \begin{tabular}[c]{@{}l@{}}Ilość elementów znajdująca się w tymczasowym buforze \\ koloru\end{tabular}                                                                                                                                                                                                                                                \\ \hline
\textbf{$\mathrm{\textbf{\scriptsize{NUM\_OF\_PIXELS}}}$}                       & \textit{(automatyczne)}              & Ilość pikseli obrazu do wyrenderowania                                                                                                                                                                                                                                                                                                                \\ \hline
\textbf{$\mathrm{\textbf{\scriptsize{TEXT\_PAGE\_SIZE}}}$}                      & $\mathtt{((unsigned)(256 \cdot 256))}$   & Ilość 32-bitowych wartości dostępnych jako pamięć tekstur                                                                                                                                                                                                                                                                                                \\ \hline
\textbf{$\mathrm{\textbf{\scriptsize{LIGHTS\_NUM}}}$}                           & $\mathtt{((unsigned\ char)2)}$       & \begin{tabular}[c]{@{}l@{}}Maksymalna ilość świateł w scenie. Pierwsze ze świateł \\ jest zawsze światłem otoczenia, pozostałe to źródła \\ punktowe\end{tabular}                                                                                                                                                                                     \\ \hline
\textbf{$\mathrm{\textbf{\scriptsize{OBJ\_NUM}}}$}                              & $\mathtt{((unsigned\ char)8)}$       & \begin{tabular}[c]{@{}l@{}}Maksymalna ilość obiektów, które mogą znaleźć się \\ w generowanej scenie\end{tabular}                                                                                                                                                                                                                                     \\ \hline
\textbf{$\mathrm{\textbf{\scriptsize{RAYTRACING\_DEPTH}}}$}                     & $\mathtt{((unsigned\ char)2)}$       & \begin{tabular}[c]{@{}l@{}}Głębokość zastosowanego śledzenia promieni. \\ Wartość równa 1 odpowiada za śledzenie jedynie promieni \\ pierwotnych - wartości ponad tę liczbę sprawiają,\\ że zaczynają być rozpatrywane odpowiednie rodziny \\ promieni wtórnych\end{tabular}                                                                          \\ \hline
\textbf{$\mathrm{\textbf{\scriptsize{MAX\_POWER\_LOOP\_ITER}}}$}                & $\mathtt{((unsigned\ char)10)}$      & \begin{tabular}[c]{@{}l@{}}Maksymalna ilość iteracji wykorzystywana w algorytmie \\ dokonującym potęgowania przez obliczanie kwadratów \\ liczb. Jednoznacznie definiuje maksymalny możliwy \\ wykładnik potęgi\end{tabular}                                                                                                                          \\ \hline
\textbf{$\mathrm{\textbf{\scriptsize{FAST\_INV\_SQRT\_ORDER}}}$}                & $\mathtt{((unsigned\ char)2)}$       & \begin{tabular}[c]{@{}l@{}}Ilość iteracji wykorzystywana przy szybkim obliczaniu \\ odwrotności pierwiastka kwadratowego\end{tabular}                                                                                                                                                                                                                 \\ \hline
\textbf{$\mathrm{\textbf{\scriptsize{FAST\_DIVISION\_ORDER}}}$}                 & $\mathtt{((unsigned\ char)2)}$       & \begin{tabular}[c]{@{}l@{}}Ilość iteracji wykorzystywana przy szybkim obliczaniu \\ ilorazu liczb.\end{tabular}                                                                                                                                                                                                                                       \\ \hline
\textbf{$\mathrm{\textbf{\scriptsize{*PI*}}}$}                                  & \textit{(różne)}                     & \begin{tabular}[c]{@{}l@{}}Zespół stałych związanych z przetwarzaniem wartości \\ liczby $\pi$\end{tabular}                                                                                                                                                                                                                                           \\ \hline
\end{longtable}
\end{landscape}