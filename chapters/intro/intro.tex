\addcontentsline{toc}{chapter}{Wstęp}
\chapter*{Wprowadzenie}
{

\pagestyle{empty}
\pagestyle{fancy}
\fancyhead{} % clear all header fields
\fancyhead[RO,LE]{\thepage}
\fancyhead[RE,LO]{Wprowadzenie}

Celem najnowszych technologii jest przede wszystkim zwiększanie jakości życia, wygody oraz produktywności korzystających z nich ludzi. Gdy określony krąg specjalistów znających się na danym zagadnieniu stara się stworzyć z użyciem własnych umiejętności rozwiązanie przystępne dla szerszego grona odbiorców mamy do czynienia z postępem.

Kiedy powstawały pierwsze próby stworzenia możliwie zrozumiałego opisu zachowania układów elektronicznych za pomocą języków opisu sprzętu, nikt nie był zainteresowany tym by upodabniać je do rozpowszechnionych sekwencyjnych języków programowania - liczyło się przede wszystkim wierne odwzorowanie sposobu, w jaki dane transferowane są pomiędzy różnymi elementami elektronicznymi. Opanowanie tych języków wiąże się z zupełnie innym postrzeganiem przetwarzania informacji. 

Obecnie okazuje się za sprawą syntezy wysokiego poziomu, takiej jak Vivado HLS, że do konfiguracji układu elektronicznego można wykorzystać wysokopoziomowe języki programowania takie jak C/C++. Dzięki temu programiści mogą wykorzystywać możliwości sprzętowej akceleracji problemów bez konieczności posiadania specjalistycznej wiedzy z zakresu języków opisu sprzętu.

Niniejsza praca koncentruje się właśnie na sprawdzeniu użyteczności narzędzia jakim jest Vivado HLS z użyciem układów FPGA firmy Xilinx na przykładzie generowania realistycznych obrazów metodą śledzenia promieni. Technika ta jest kosztowna obliczeniowo za pomocą tradycyjnych technik opartych o wykonywanie algorytmu przez procesor komputera, jednak w przeciwieństwie do rastryzacji daje realistyczne obrazy w sposób bezpośredni, tylko na podstawie analizy rozchodzenia się światła w przestrzeni.

Niniejsza praca została podzielona na 3 rozdziały:
\begin{enumerate}
\item Opisuje, na czym polega technika śledzenia promieni, jakie podstawowe prawa i koncepcje fizyczne są przez nią wykorzystywane. Przedstawione tu zostały również pewne modele służące opisaniu zachowania się różnych typów powierzchni w momencie interakcji ze światłem.
\item Pokazuje, dlaczego układy FPGA są ciekawym narzędziem służącym do akceleracji przetwarzania danych. Opisane zostały cechy Vivado HLS odróżniające go od zwykłych języków sekwencyjnych~(w tym dyrektywy optymalizacyjne) oraz fundamentalne informacje dotyczące  implementacji stworzonego przy pomocy Vivado HLS modułu w strukturze układu FPGA.
\item Rozdział ten podaje, w jaki sposób próbowano stworzyć moduł dokonujący akceleracji śledzenia promieni przy pomocy Vivado HLS. Pokazuje on wady i zalety tego narzędzia w różnych sytuacjach, a także opisuje możliwości finalnego modułu, który jest implementowalny w strukturze układu FPGA. 
\end{enumerate}
}
%Lorem ipsum\newpage fasdfs
%\cite{Hecht}\cite{PBRT}\cite{RTFTGU}\cite{OPENGL46}
